\documentclass[11pt]{article}
\usepackage[a4paper,margin=0.53in]{geometry}

\usepackage{enumitem}
\usepackage{hyperref}
\usepackage{algorithm}
\usepackage{algpseudocode}
\usepackage{multicol}
\usepackage{amsmath}
\usepackage{amssymb}
\usepackage[xetex]{graphicx}
\usepackage[pdf]{graphviz}
\usepackage{listings}
\usepackage{xcolor}
\usepackage{relsize}
\usepackage{lipsum}
\usepackage{booktabs}
\usepackage{fontspec}
\usepackage{hyperref}
\setmainfont{Alegreya}

%\renewcommand{\familydefault}{\sfdefault}
\lstset{basicstyle=\ttfamily\footnotesize,breaklines=true}
\lstset{numbers=left,xleftmargin=2em,framexleftmargin=4em}

%\graphicspath{}

\newcommand{\slide}[1]{\textbf{Slide \##1}}
\newcommand\numberthis{\addtocounter{equation}{1}\tag{\theequation}}
\newcommand{\lstpartialinputlisting}[3]{\lstinputlisting[firstnumber=#2,firstline=#2,lastline=#3]{#1}}


\setlength\parindent{0pt}

%\twocolumn

\title{Recent developments around Section 377}
\author{Sathyam Vellal}
\date{\today\\(\footnotesize Created on January 10, 2018)}

\begin{document}
\maketitle

10,000 ft view -
\begin{enumerate}[noitemsep]
	\item Section 377 is a victorian era law criminalising homosexuality in India. (technically, "sex against the order of nature"). 
	\item The Delhi High Court struck down this law, decriminalising in 2009.
	\item The Supreme Court reversed the decision in 2013, saying it was a matter to be taken up by the parliament (where it never saw the light of day).
	\item Last year (2017), the Supreme Court asserted (in a landmark case) that "right to privacy" is a fundamental right protected by the Indian constitution. (It even went ahead to explicitly state "sexual orientation" in the list of examples it provided in the judgement.)
	\item On January 8, 2018, a three-judge bench reviewed the 2013 ruling, against the 2017 ruling and said that the validity of Section 377 needs to be looked afresh. (edited)
\end{enumerate}

Right to Privacy Judgement of 2017 - 
\begin{enumerate}[noitemsep]
	\item When large scale Aadhar adoption started around 2014-2015, many people were concerned by the collection of biometric data. 
	\item Some activists then approached the Supreme Court with concerns over privacy.
	\item A quick survey revealed that "Right to Privacy" was something that wasn't explicit in the constitution of India, and that it needs to be examined.
	\item A constitution bench was formed (a group of at least five judges who assemble specifically to interpret the constitution, aka. the law) to examine this issue.
	\item After a couple of months, a nine-judge constitution bench of the Supreme Court unanimously ruled that "Right to Privacy" was a fundamental right protected by the Constitution of India.
\end{enumerate}

In the ruling, the Supreme Court, among many things, said -\\
{\itshape Sexual orientation is an essential attribute of privacy. Discrimination against an individual on the basis of sexual orientation is deeply offensive to the dignity and self-worth of the individual. Equality demands that the sexual orientation of each individual in society must be protected on an even platform. The right to privacy and the protection of sexual orientation lie at the core of the fundamental rights guaranteed by Articles 14, 15 and 21 of the Constitution.}

Why did the Supreme Court address sexual orientation as well, specifically? 
\begin{enumerate}[noitemsep]
	\item Apparently, the justices in the Supreme Court themselves were torn at reversing the Delhi High Court's 2009 verdict, in 2013. 
	\item  The United Nations also condemned the decision, and many human rights activists, equal rights activists, organisations/foundations have been constantly criticising it. Not to mention the amount of media attention it got (because if Supreme Court allowed Section 377 to fall, it'd be a major to victory to the world. Thanks to India's population, it's still seen as a major player in Asian social and economic circles)
	\item The judges apparently have been looking for opportunities for bringing up this issue back on table. 
	\item Among the petitions filed to be considered what falls under the umbrella of Right to Privacy, sexual orientation was also submitted for consideration.
\end{enumerate}

The activists' narrative -
\begin{enumerate}[noitemsep]
	\item The Supreme Court said in 2013 that Section 377 is almost never enforced for cases of homosexuality. For 150 year's it's been up, around 1300 cases have been registered, and almost all, mostly for sexual abuse and paedophilia cases
	\item Activists said, while that's true, just by that law being in place instills life-long fear in the sexual minorities, and hence this has to be addressed.
	\item They got the nine-judge bench to express what the Constitution of India says about "sexual orientation". 
	\item They then petitioned the Supreme Court again requesting it to reconsider the 2013 verdict on Section 377.
\end{enumerate}

Supreme Court reconsidering 2013 verdict -
\begin{enumerate}[noitemsep]
	\item A nine-judge bench last year ruled saying that Right to Privacy is a fundamental right and protected by the Indian constitution. 
	\item A three-judge bench was formed to look into whether the Supreme Court needs to reconsider the 2013 verdict. 
	\begin{enumerate}[noitemsep]
		\item Regarding sex and sexual orientation - The determination of "order of nature" is not a constant phenomenon.
		\item Regarding morality - Societal morality changes with time, law walks with life.
		\item Regarding choice of sexual partner - A section of people can't live in fear of their individual choice.
	\end{enumerate}
	\item And by this, it said that it rightfully needs to reconsider the 2013 verdict.
\end{enumerate}

\begin{enumerate}[noitemsep]
	\item Section 377 is practically dead right now. It'll be struct down formally very soon.
	\item FAQs
	\begin{enumerate}[noitemsep]
		\item Regarding his "practically dead claim" - With the Right to Privacy judgement from last year, this can be answered. The nine-judge bench ruled that Privacy is a fundamental right. Hence, there's no question about homosexuality being legal. It can only be contested if an eleven-judge bench steps in and overturns that ruling. Hence, in a way, Section 377, is dead. This just needs to be made official through formal processes.
		\item Regarding what the centre can do - That ship has sailed. The nine-judge bench has discussed and stated what the constitution says regarding privacy as a fundamental right. It's now a matter of quoting those four paragraphs from the nine-judge bench and striking down Section 377. Whatever views the political class may have, it's too late for them now. They can't even amend the constitution to change this.
		\item Regarding 2013 ruling - Back in 2013, the real question to have been answered was the right to privacy. The Supreme Court only said, that it wasn't for the courts to decide, but a constitutional matter. The decision is justified. But it certainly was flawed.
	\end{enumerate}
\end{enumerate}

Prominent LGBT rights activists - 
\begin{enumerate}[noitemsep]
	\item Anjali Gopalan. Founder and present director of the Naz Foundation, an NGO, which was set up to work on HIV/AIDS and sexual health
	\item Aditya Bandopadhyay, an LGBT activist and lawyer.
	\item Gauri Sawant, the transgender from the extremely popular Vicks ad.
	\item Laxmi Narayan Tripathi, a transgender activist
	\item Ashok Row Kavi, one of the first people to openly talk about homosexuality. Ahsoka India fellow.
	\item Harish Iyer, who's played an important role behind the scenes and behind the camera in almost everything, including the Satyameva Jayate series. Occasionally, he shows his face
\end{enumerate}

\subsubsection*{References}
\begin{enumerate}
	\item \href{http://www.thehindu.com/news/national/article19551816.ece/BINARY/RightToPrivacyVerdict}{Right to Privacy judgement of 2017, \lstinline{http://www.thehindu.com/news/national/article19551816.ece/BINARY/RightToPrivacyVerdict}}
\end{enumerate}
\end{document}